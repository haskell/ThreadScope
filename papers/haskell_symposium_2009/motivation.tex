\section{Profiling Motivation}
Show examples of semi-explicit parallel programs that go wrong. Show what we could measure before using heap and time profiling and motivate the need for better profiling.

Haskell provides a mechanism to allow the user to control the granularity of parallelism by indicating what computations may be usefully carried out in parallel. This is done by using functions from the \codef{Control.Parallel} module. The interface for \codef{Control.Parallel} is shown below:
\begin{lstlisting}
  par :: a -> b -> b 
  pseq :: a -> b -> b 
\end{lstlisting}
The function \codef{par} indicates to the GHC run-time system that it may be beneficial to evaluate the first argument in parallel with the second argument. The \codef{par} function returns as its result the value of the second argument. One can always eliminate \codef{par} from a program by using the following identity without altering the semantics of the program:
\begin{lstlisting}
  par a b = b 
\end{lstlisting}
A thread is not necessarily created to compute the value of the expression \codef{a}. Instead, the GHC run-time system creates a {\em spark} which has the potential to be executed on a different thread from the parent thread. A sparked computation expresses the possibility of performing some speculative evaluation. Since a thread is not necessarily created to compute the value of \codef{a} this approach has some similarities with the notion of a {\em lazy future}~\cite{mohr:91}.

Sometimes it is convenient to write a function with two arguments as an infix function and this is done in Haskell by writing quotes around the function:
\begin{lstlisting}
  a `par` b
\end{lstlisting}

We call such programs semi-explicitly parallel because the programmer has provided a hint about the appropriate level of granularity for parallel operations and the system implicitly creates threads to implement the concurrency. The user does not need to explicitly create any threads or write any code for inter-thread communication or synchronization.

To illustrate the use of \codef{par} we present a program that performs two compute intensive functions in parallel. The first compute intensive function we use is the notorious Fibonacci function:
\begin{lstlisting}
fib :: Int -> Int
fib 0 = 0
fib 1 = 1
fib n = fib (n-1) + fib (n-2)
\end{lstlisting}
The second compute intensive function we use is the \codef{sumEuler} function taken from~\cite{trinder:02}:
\begin{lstlisting}
mkList :: Int -> [Int]
mkList n = [1..n-1]

relprime :: Int -> Int -> Bool
relprime x y = gcd x y == 1

euler :: Int -> Int
euler n = length (filter (relprime n) (mkList n))

sumEuler :: Int -> Int
sumEuler = sum . (map euler) . mkList
\end{lstlisting}
The function that we wish to parallelize adds the results of calling \codef{fib} and \codef{sumEuler}:
\begin{lstlisting}
sumFibEuler :: Int -> Int -> Int
sumFibEuler a b = fib a + sumEuler b
\end{lstlisting}
As a first attempt we can try to use \codef{par} the speculatively spark off the computation of \codef{fib} while the parent thread works on \codef{sumEuler}:
\begin{lstlisting}
-- A wrong way to parallelize f + e
parSumFibEuler :: Int -> Int -> Int
parSumFibEuler a b
  = f `par` (f + e)
    where
    f = fib a
    e = sumEuler b
\end{lstlisting}

To create two workloads that take roughly the same amount of time to execute we performed some experiments which show that \codef{fib 38} takes roughly the same time to execute as \codef{sumEuler 5300}. If we were to run this program and view the execution trace we would see somthing like the graph shown in Figure~\ref{f:wrongpar}.

\begin{figure}
\begin{center}
\includegraphics[width=8.5cm]{SumEuler1-N2-eventlog.pdf}
\end{center}
\caption{No parallelization of \codef{f `par` (f + e)}}
\label{f:wrongpar}
\end{figure}
